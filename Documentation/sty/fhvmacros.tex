\newcommand{\getAuthor}{AUTHOR}
\newcommand{\setAuthor}[1]{\renewcommand{\getAuthor}{#1}}

\newcommand{\getAuthorId}{Matrikelnummer}
\newcommand{\setAuthorId}[1]{\renewcommand{\getAuthorId}{#1}} 

\newcommand{\getSupervisor}{1234}
\newcommand{\setSupervisor}[1]{\renewcommand{\getSupervisor}{#1}} 

\newcommand{\getTitle}{TITLE}
\newcommand{\setTitle}[1]{\renewcommand{\getTitle}{#1}}

\newcommand{\getInstituteLocation}{Dornbirn, Austria}
\newcommand{\setInstituteLocation}[1]{\renewcommand{\getInstituteLocation}{#1}}

\newcommand{\getInstituteGraphic}{\includegraphics[width = 0.5 \textwidth]{./img/FHVlogo}}
\newcommand{\setInstituteGraphic}[1]{\renewcommand{\getInstituteGraphic}{#1}}

\renewcommand{\maketitle}{
\begin{titlepage}
\begin{center}

% Upper part of the page. The '~' is needed because \\
% only works if a paragraph has started.

\textsc~\getInstituteGraphic
\\[1.5cm]

\textsc{\Large Bachelorarbeit über das Berufspraktikum im Rahmen des Bachelor Studienganges Informatik  - Software and Information Engineering}\\[1.5cm]
%Title
\hrule

\vspace{0.5cm}
\textsc{\huge \bfseries \getTitle}\\[0.5cm]

\hrule
\vspace{1.5cm}

% Author and supervisor
\textsc{verfasst von}
\\[0.4cm]

\textsc{\Large \getAuthor}
\\
\getAuthorId
\\[1cm]


\textsc{Betreut durch}
\vspace{0.4cm}

\textsc{\Large \getSupervisor}

\vspace{1.5cm}

\textsc{durchgeführt bei proTask GmbH betreut durch}
\vspace{0.4cm} 
\\ 
\textsc{\Large Andreas Saler}

\vspace{1.5cm}

% Bottom of the page
\textsc{Dornbirn, \today}

\end{center}
\end{titlepage}
}

\newcommand{\newAcronymAndGlossaryEntry}[4]
{
\newacronym[description={\glslink{#2}{#3}}]
{#1}{#2}{#3}

\newglossaryentry{#2}{name=\glslink{#1}{#3},
text=#3,
description=#4}
}

\newcommand{\glossaryAndBibliography}{

% Table of Illustrations
\renewcommand\listfigurename{Abbildungen}
\listoffigures
\addcontentsline{toc}{chapter}{Abbildungen}

% Anything following the further research must not have a chapternumber
\renewcommand{\chaptername}{}
\renewcommand{\thechapter}{}

%Glossary
%\addcontentsline{toc}{chapter}{Glossary}
%\printglossaries

\printglossary[type=\acronymtype] % prints just the list of acronyms

%\glsaddall

%\addcontentsline{toc}{chapter}{Glossary}
\printglossary % if no option is supplied the default glossary is printed.

% Bibliography
\clearpage
\phantomsection
\addcontentsline{toc}{chapter}{Literaturverzeichnis}
\bibliography{./bib/mybibtexfile}{}
}